\documentclass[10pt,a4paper]{report}
\usepackage[utf8]{inputenc}
\usepackage{amsmath}
\usepackage{amsfonts}
\usepackage{amssymb}
\author{Jacek Czyrnek, Adam Koleszar, Marion Le Guével, Mihaly Lengyel
•
}
\usepackage{graphicx}
\graphicspath{{/home/pandachan/Documents/cours/Group_project/}}
\title{Group project report}
\begin{document}

\chapter{Introduction}


\chapter{Technical review}


\chapter{Requirements specifications}
	\subsection{Usecases}
This project handle login as sessions. Therefore each session (a set of iterations) is a new user with a new password.  However, all these "users" are of the same type and have the same functionnalities : \\
\includegraphics[scale=0.4]{Usecase_group-project.png}
Mainly, the purpose of this software is to solve an equation with different given parameters and to display the result graphically.
	\subsection{Functional and non functional requirements}
On the previous usecase, different functionalities are written. This subsection will detailed those functionalities into functional and non functional requirements.
		\subsubsection{Functional requirements}
\begin{tabular}{|c|c|}
\hline 
\textbf{ID} & \textbf{Requirement} \\ 
\hline 
1 & To log into a session \\ 
\hline 
2 & To create a new session \\ 
\hline 
3 & As logged in : To enter new inputs \\ 
\hline
4 & As logged in : To start or stop iterating on the equation (exchange of data with the cluster) \\ 
\hline
5 & As logged in : To display the result of the iterations in real time on a graph \\ 
\hline
6 & As logged in : To display the result of the iterations in real time as a wing modeling \\ 
\hline
7 & As logged in : To download the log file (exchange of data with the cluster)\\ 
\hline
8 & As logged in : To logout from a session \\ 
\hline
\end{tabular} 
		\subsubsection{Non-functional requirements}
\begin{tabular}{|c|c|}
\hline 
\textbf{Type} & \textbf{Requirement} \\ 
\hline 
Security & To log into a session with a password so only people having those\\
 & information can access to it\\ 
\hline 
Reliability & To have a recovery system to access previous steps of a session \\ 
\hline 
Speed & To use a cluster for solving the equation \\ 
\hline
\end{tabular} 
	\subsection{Further information}
An example has been given for the software user interface. With the functionalities required, the final product should have two "pages", one for the connexion or creation of a session, and another one once logged in a session, as the mockups below:\\
\includegraphics[scale=0.4]{group_project_mockups.png}
\end{document}